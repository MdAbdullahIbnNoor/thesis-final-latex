\RequirePackage[l2tabu]{nag}		% Warns for incorrect (obsolete) LaTeX usage

\documentclass[a4paper,12pt,leqno,openbib,oneside]{memoir} 
\usepackage{gensymb}
\usepackage{datetime}
\usepackage{ifpdf}
\usepackage{geometry}
\usepackage{booktabs}
\usepackage{adjustbox}
\usepackage{mathptmx} % Times New Roman-like font
\usepackage[T1]{fontenc}

% Set margins: 1.5in left, 1in top/right/bottom
\geometry{left=1.5in, right=1in, top=1in, bottom=1in}

\ifpdf
\pdfinfo{
   /Author (Mrs. Shaima Aslam Chaity)
   /Title (Undergrad Thesis)
   /Keywords (200121)
   /CreationDate (D:\pdfdate)
}
\fi
 
\ifdraftdoc 
	\usepackage{draftwatermark}				
	\SetWatermarkScale{0.3}
	\SetWatermarkText{\bf Draft: \today}
\fi

% Declare figure/table as a subfloat.
\frenchspacing
% Set 1.5 line spacing globally
\OnehalfSpacing 
% Sets numbering division level
\setsecnumdepth{subsection} 
\maxsecnumdepth{subsubsection}

% Chapter style (taken and slightly modified from Lars Madsen Memoir Chapter Styles)
\usepackage{calc,soul,fourier}
\makeatletter 
\newlength\dlf@normtxtw 
\setlength\dlf@normtxtw{\textwidth} 
\newsavebox{\feline@chapter} 
\newcommand\feline@chapter@marker[1][4cm]{%
	\sbox\feline@chapter{% 
		\resizebox{!}{#1}{\fboxsep=1pt%
			\colorbox{gray}{\color{white}\thechapter}% 
		}}%
		\rotatebox{90}{% 
			\resizebox{%
				\heightof{\usebox{\feline@chapter}}+\depthof{\usebox{\feline@chapter}}}% 
			{!}{\scshape\so\@chapapp}}\quad%
		\raisebox{\depthof{\usebox{\feline@chapter}}}{\usebox{\feline@chapter}}%
} 
\newcommand\feline@chm[1][4cm]{%
	\sbox\feline@chapter{\feline@chapter@marker[#1]}% 
	\makebox[0pt][c]{% aka \rlap
		\makebox[1cm][r]{\usebox\feline@chapter}%
	}}
\makechapterstyle{daleifmodif}{
	\renewcommand\chapnamefont{\normalfont\Large\scshape\raggedleft\so} 
	\renewcommand\chaptitlefont{\normalfont\Large\bfseries\scshape} 
	\renewcommand\afterchapternum{\par\vskip\midchapskip} 
}
\makeatother 
\chapterstyle{daleifmodif}

% Page style: page numbers at bottom center, headers for sections
\makepagestyle{myvf} 
\makeoddfoot{myvf}{}{\thepage}{} 
\makeevenfoot{myvf}{}{\thepage}{} 
\makeheadrule{myvf}{\textwidth}{\normalrulethickness} 
\makeevenhead{myvf}{\small\textsc{\leftmark}}{}{} 
\makeoddhead{myvf}{}{}{\small\textsc{\rightmark}}
\pagestyle{myvf}

% Command to clear to next odd page
\newcommand{\clearemptydoublepage}{\newpage{\thispagestyle{empty}\cleardoublepage}}

% Creates indexes for Table of Contents, List of Figures, List of Tables, and Index
\makeindex

\usepackage{import}
\usepackage{lipsum}
\usepackage{amsfonts}
\usepackage[centertags]{amsmath}
\usepackage{stmaryrd}
\usepackage{amssymb}
\usepackage{enumitem}
\usepackage{amsthm}
\usepackage{newlfont}
\usepackage{graphicx}
\usepackage{longtable,rotating}
\usepackage[applemac]{inputenc}
\usepackage{colortbl}
\usepackage{wasysym}
\usepackage{mathrsfs}
\usepackage{subcaption}
\usepackage{float}
\usepackage{multirow}
\usepackage{verbatim}
\usepackage{upgreek}
\usepackage{latexsym}
\usepackage[square,numbers,sort&compress]{natbib}
\usepackage{url}
\usepackage[spanish,english]{babel}
\usepackage{color}
\usepackage[colorlinks=true,allcolors=black]{hyperref}
\usepackage{memhfixc}
\usepackage{footnote}
\usepackage{microtype}
\usepackage{rotfloat}
\usepackage{alltt}
\usepackage[version=0.96]{pgf}
\usepackage{tikz}
\usepackage{graphicx} % For figure environment
\usepackage{float}
\usepackage{seqsplit}
\usetikzlibrary{arrows,shapes,automata,backgrounds,petri,topaths, shapes.geometric, arrows.meta}

% Reduce widows and orphans
\widowpenalty=1000
\clubpenalty=1000

% Custom commands
\newcommand{\keywords}[1]{\par\noindent{\small{\bf Keywords:} #1}}
\newcommand{\parcial}[2]{\frac{\partial#1}{\partial#2}}
\newcommand{\vectorr}[1]{\mathbf{#1}}
\newcommand{\vecol}[2]{\left(\begin{array}{c} \displaystyle#1 \\ \displaystyle#2 \end{array}\right)}
\newcommand{\mados}[4]{\left(\begin{array}{cc} \displaystyle#1 &\displaystyle #2 \\ \displaystyle#3 & \displaystyle#4 \end{array}\right)}
\newcommand{\pgftextcircled}[1]{%
    \setbox0=\hbox{#1}%
    \dimen0\wd0%
    \divide\dimen0 by 2%
    \begin{tikzpicture}[baseline=(a.base)]%
        \useasboundingbox (-\the\dimen0,0pt) rectangle (\the\dimen0,1pt);
        \node[circle,draw,outer sep=0pt,inner sep=0.1ex] (a) {#1};
    \end{tikzpicture}
}
\newcommand{\range}[1]{\textnormal{range }#1}
\newcommand{\innerp}[2]{\left\langle#1,#2\right\rangle}
\newcommand{\prom}[1]{\left\langle#1\right\rangle}
\newcommand{\tra}[1]{\textnormal{tra} \: #1}
\newcommand{\sign}[1]{\textnormal{sign\,}#1}
\newcommand{\sech}[1]{\textnormal{sech} #1}
\newcommand{\diag}[1]{\textnormal{diag} #1}
\newcommand{\arcsech}[1]{\textnormal{arcsech} #1}
\newcommand{\arctanh}[1]{\textnormal{arctanh} #1}
\newcommand{\blackged}{\hfill$\blacksquare$}
\newcommand{\whiteged}{\hfill$\square$}
\newcounter{proofcount}
\renewenvironment{proof}[1][\proofname.]{\par
 \ifnum \theproofcount>0 \pushQED{\whiteged} \else \pushQED{\blackged} \fi%
 \refstepcounter{proofcount}
 \normalfont 
 \trivlist
 \item[\hskip\labelsep \itshape {\bf\em #1}]\ignorespaces
}{\addtocounter{proofcount}{-1} \popQED\endtrivlist}

% New square root definition
\let\oldsqrt\sqrt
\def\sqrt{\mathpalette\DHLhksqrt}
\def\DHLhksqrt#1#2{%
\setbox0=\hbox{$#1\oldsqrt{#2\,}$}\dimen0=\ht0
\advance\dimen0-0.2\ht0
\setbox2=\hbox{\vrule height\ht0 depth -\dimen0}%
{\box0\lower0.4pt\box2}}

% Caption styles
\newcommand{\mycaption}[2][\@empty]{%
	\captionnamefont{\scshape} 
	\changecaptionwidth
	\captionwidth{0.9\linewidth}
	\captiondelim{.\:} 
	\indentcaption{0.75cm}
	\captionstyle[\centering]{}
	\setlength{\belowcaptionskip}{10pt}
	\ifx \@empty#1 \caption{#2}\else \caption[#1]{#2}
}
\newcommand{\mysubcaption}[2][\@empty]{%
	\subcaptionsize{\small}
	\hangsubcaption
	\subcaptionlabelfont{\rmfamily}
	\sidecapstyle{\raggedright}
	\setlength{\belowcaptionskip}{10pt}
	\ifx \@empty#1 \subcaption{#2}\else \subcaption[#1]{#2}
}

% Initial letter
\usepackage{lettrine}
\newcommand{\initial}[1]{%
	\lettrine[lines=3,lhang=0.33,nindent=0em]{\color{gray}{\textsc{#1}}}{}}

% Theorem styles
\theoremstyle{plain}
\newtheorem{theo}{Theorem}[chapter]
\theoremstyle{plain}
\newtheorem{prop}{Proposition}[chapter]
\theoremstyle{plain}
\theoremstyle{definition}
\newtheorem{dfn}{Definition}[chapter]
\theoremstyle{plain}
\newtheorem{lema}{Lemma}[chapter]
\theoremstyle{plain}
\newtheorem{cor}{Corollary}[chapter]
\theoremstyle{plain}
\newtheorem{resu}{Result}[chapter]

% Hyphenation
\hyphenation{res-pec-tively}
\hyphenation{mono-ti-ca-lly}
\hyphenation{hypo-the-sis}
\hyphenation{para-me-ters}
\hyphenation{sol-va-bi-li-ty}

% new things added:

% Set no paragraph indentation and increase word spacing with justification
\setlength{\parindent}{0pt} % No indentation for paragraphs
\fontdimen2\font=0.3em % Nominal inter-word space
\fontdimen3\font=0.15em % Stretch for inter-word space
\fontdimen4\font=0.1em % Shrink for inter-word space
% \setlength{\spaceskip}{0.2em}
\setbeforesecskip{4.5ex plus 1ex minus 0.2ex} % Increases space before \section
\setbeforesubsecskip{4.25ex plus 1ex minus 0.2ex} % Increases space before \subsection

\setlength{\parskip}{12pt}

\begin{document}

\frontmatter
\pagenumbering{roman}
\setlength{\parskip}{0pt}
%
%\vspace{5cm}
\begin{titlingpage}
\begin{SingleSpace}
\calccentering{\unitlength} 
\begin{adjustwidth*}{\unitlength}{-\unitlength}
%\vspace*{13mm}
%\vspace{5cm}

\begin{center}
    {\textbf{Genetic and Molecular Perspective on Endocrine and Cardiovascular Complications in Beta-Thalassemia Patients}}
\end{center}

\vspace{0.5cm}

\begin{center}
\includegraphics[width=32mm]{logos/pust_logo.jpg} %Engelsk versjon
%\includegraphics[width=140mm]{Figures/uia-logo-nor.pdf} %Norsk versjon

Department of Computer Science and Engineering

Pabna University of Science and Technology, Pabna-6600

\vspace{0.5cm}

Course Title: Thesis

Course Code: CSE 4100 and CSE 4200


\vspace{0.5cm}

{\em \textbf{A thesis has been submitted to the Department of Computer Science and Engineering for the partial fulfillment of the requirement of B.Sc in Engineering in Computer Science and Engineering}
}

\vspace{1cm}
{\textbf {Submitted By:}}

The Examinee of B.Sc Engineering Final Examination-2022

Mst. Shaima Ashlam Chaity

Roll Number: 200121

Registration Number: 1011879

Session: 2019-20

\vspace{1cm}

{\textbf{Supervised By:}}

{\textbf{S. M. Hasan Sazzad Iqbal}}

Associate Professor

Department of Computer Science and Engineering

Pabna University of Science and Technology

\vspace{1cm}

\textbf{Laboratory}

Advanced Computer Lab, Department of Computer Science and Engineering

Pabna University of Science and Technology

\vspace{0.5 cm}

{\textbf{August, 2024 }}

\end{center}


%\textsc
%\clearpage

%\begin{flushright}
%{\small Word count: ten thousand and four}
%\end{flushright}
\end{adjustwidth*}
\end{SingleSpace}
\end{titlingpage}
\setlength{\parskip}{0pt}

\chapter*{\begin{center}
{\Large \bfseries DECLARATION}\\[1cm]
\end{center}}



\begin{SingleSpace}
\begin{quote}

%\vspace{1mm}

\large {
I, \textbf{Shaima Aslam Chaity}, hereby declare that the work presented in this thesis is entitled \textbf{``Genetic and Molecular Perspective on Endocrine and Cardiovascular Complications in Beta-Thalassemia Patients``}, is the outcome of my own research and effort carried out under the supervision of \textbf{S. M. Hasan Sazzad Iqbal}, Department of Computer Science and Engineering, Pabna University of Science and Technology (PUST), Pabna.

\vspace{0.5cm}

I also declare that this work is the best of my knowledge and does not contain any material that previously published or written by another person, nor it has been submitted in whole or in part, for the award of any degree or diploma at any other university or institution. All materials or content taken from other sources has been appropriately referenced in this thesis

\vspace{0.5cm}

This thesis reflects my own findings, opinions and conclusions and do not necessarily represent the views of Pabna University of Science and Technology or any other institution.
}

\vspace{1cm}

\vspace{15mm}
 
%\vspace{4cm}

%\vspace{25mm}



\begin{center}

\begin{flushright}
 
\line(1,0){150}\\

{\textbf{Signature of the Examinee}}

 



 \end{flushright}
%\large{Signature of the Supervisor}\\

\end{center}

\end{quote}
\end{SingleSpace}
%\clearpage

\setlength{\parskip}{0pt}
\chapter*{\begin{center}
{\Large \bfseries CERTIFICATION}\\[1cm]
\end{center}}
\begin{quote}
\vspace{2mm}

\large {I am happy to certify that Mst. Shaima Ashlam Chaity, Roll Number: 190116, Registration Number: 101832, Session: 2018-2019, has completed a thesis work that enabled \textbf{``A Hybrid Methodology for Recognizing Bangla Sign Language Using A Deep Transfer Learning Model in Combination with A Machine Learning Classifier''} under my supervision to fulfill the requirements of the thesis course. This thesis was completed for a year at the Department of Computer Science and Engineering, Pabna University of Science and Technology, Pabna-6600, Bangladesh.

\vspace{0.5cm}

According to my knowledge, this thesis paper has not been submitted elsewhere or replicated by another thesis paper before being submitted to the department.
} 
\vspace{1cm}

\vspace{10mm}

\vspace{2mm}

\begin{flushleft}
 
{\textbf{S. M. Hasan Sazzad Iqbal}}
 
Associate Professor,

Department of Computer Science and Engineering

Pabna University of Science and Technology, Pabna, Bangladesh


\end{flushleft}
\end{quote}
%\clearpage


\chapter*{\begin{center}
{\Large \bfseries ACKNOWLEDGEMENT}
\end{center}}

\vspace{1cm}
%\begin{SingleSpace}
My deepest gratitude to the Allah for granting me the strength and guidance to complete this thesis successfully.
\vspace{3mm}

I am sincerely thankful to my supervisor, \textbf{S. M. Hasan Sazzad Iqbal}, Associate Professor in Department of Computer Science and Engineering at Pabna University of Science and Technology (PUST), for his continuous support, insightful advice and valuable feedback throughout the completion of this research. His insightful suggestions, encouragement, and guidance have been instrumental in shaping this work from its beginning to completion.

\vspace{3mm}

I would also like to extend my appreciation to all the respected teachers of the Department of Computer Science and Engineering, PUST, whose teachings and academic guidance have laid the foundation for my research knowledge and skills. My heartfelt thanks go to my family, especially my parents as well as my friends and well-wishers for their support, motivation, and cooperation during this work.

\begin{flushleft}

I express my genuine thanks to everyone who contributed by any means to the successful ending of this thesis.

\end{flushleft}

\vspace{3cm}

\vspace{15mm}

\begin{flushright}
\large{August, 2025}\\
\large{Author}\\
\end{flushright}
%\end{SingleSpace}
\clearpage

\chapter*{\begin{center}
{\Large \bfseries ABSTRACT}
\end{center}}
Beta-thalassemia (BT), a genetic blood anomaly brought on by abnormalities in the HBB gene that occurs when beta-globin chain production is reduced or absent in blood. It is a major worldwide health issue because of its complicated clinical management and lifetime reliance on blood transfusions. Some recent evidence highlights that beta-thalassemia is associated with several comorbidities, including endocrine diseases involving polycystic ovarian syndrome (PCOS), hypothyroidism, hypogonadism and type 2 diabetes (T2D) and cardiovascular diseases involving arrhythmogenic cardiomyopathy (ACM) and arrhythmia. These comorbidities may share common molecular mechanisms with beta-thalassemia that potentially exacerbate disease severity and treatment complications. In this study, a computational approach was applied to evaluate the genetic relationships between beta-thalassemia and its associated comorbidities using microarray and mRNA datasets that are publicly available in NCBI. Genetic profiling was constructed to identify the common matching genes and built disease-gene networks (DGNs) of matching genes. It also visualized a heatmap to present patterns of gene expressions. We also explored multiple bioinformatics analysis including pathways, gene ontology, protein-protein interaction (PPI) and protein-drug interaction (PDI) that strongly indicate their correlation. A validation network was created to verify our selected comorbidity and then phylogenetic analysis was performed for all diseases to determine their evolutionary relationships. This study found that beta-thalassemia shares 13, 165, 13, 14, 11 and 44 significantly expressed genes with hypothyroidism, hypogonadism, PCOS, T2D, ACM and arrhythmia respectively. The outcomes of this study may help in integrative medical approaches and enhance a significant understanding of genetic and molecular structure of comorbidities in beta-thalassemia by providing valuable insights.

\vspace{0.5cm}
 
\textbf{Keywords}: Beta-thalassemia, comorbidities, genetic profiling, endocrine diseases, cardiac diseases, arrhythmogenic cardiomyopathy, pathway analysis, ontology, phylogenetic analysis.

\clearpage


\renewcommand{\contentsname}{Table of Contents}
\maxtocdepth{subsection}
\tableofcontents*
\addtocontents{toc}{\par\nobreak \mbox{}\hfill{\bf Page}\par\nobreak}
\clearpage

\chapter*{Abbreviations and Acronyms}
\addcontentsline{toc}{chapter}{Abbreviations and Acronyms}  % Adds to Table of Contents

\begin{SingleSpace}
\large The following abbreviations and acronyms are used in this thesis: 
\vspace{5mm}

\begin{itemize}
    \item \textbf{KU-BdSL}: Khulna University Bengali Sign Language dataset
    \item \textbf{BdSL}: Bangla Sign Language
    \item \textbf{WHO}: World Health Organization
    \item \textbf{SHAP}: SHapley Additive exPlanation
    \item \textbf{SE}: Squeeze Excitation
    \item \textbf{ML}: Machine Learning
    \item \textbf{DL}: Deep Learning
    \item \textbf{CNN}: Convolution Neural Network
    \item \textbf{ResNet}: Residual Neural Network
    \item \textbf{RGB}: Red, Green, and Blue
    \item  \textbf{MSLD}: Multi-scale Sign Language Dataset
    \item  \textbf{USLD}: Uni-scale Sign Language Dataset
    \item  \textbf{AMSLD}: Annotated Multi-scale Sign Language Dataset
    \item \textbf{ROC}: Receiver Operating Characteristics
\end{itemize}

\end{SingleSpace}
\clearpage

\clearpage

\listoftables
\addtocontents{lot}{\par\nobreak\textbf{{\scshape Table} \hfill Page}\par\nobreak}
\clearpage

\listoffigures
\addtocontents{lof}{\par\nobreak\textbf{{\scshape Figure} \hfill Page}\par\nobreak}
\clearpage

\mainmatter
\import{chapters/chapter01/}{chap01.tex}
\import{chapters/chapter02/}{chap02.tex}
\import{chapters/chapter03/}{chap03.tex}
\import{chapters/chapter04/}{chap04.tex}
\import{chapters/chapter05/}{chap05.tex}

\renewcommand\bibname{\fontsize{16pt}{1pt} \selectfont \centerline{REFERENCES} \global\def\bibname{REFERENCES}}

\begin{thebibliography}{99}
\bibitem{b1} Malik, S.E., et al., 2023. Statistical analysis of 135 Beta-Thalassemia Major patients. Journal of Hematology, 28(1), pp. 45-53.
\bibitem{b2} Akiki, N., et al., 2023. Patterns and mechanisms in beta-thalassemia. Clinical Reviews, 15(2), pp. 89-97.
\bibitem{b3} Podder, N.K., et al., 2020. Genetic links between Type 2 diabetes and comorbidities. Bioinformatics, 36(5), pp. 123-130.
\bibitem{b4} Podder, N.K., et al., 2021. Molecular patterns in COVID-19 and neurodegenerative diseases. Journal of Computational Biology, 28(3), pp. 210-220.
\bibitem{b5} M. S. Islam, M. M. Rahman, M. H. Rahman, M. Arifuzzaman, R. Sassi, and M. Aktaruzzaman, "Recognition Bangla sign language using convolutional neural network," in 2019 International Conference on Innovation and Intelligence for Informatics, Computing, and Technologies (3ICT), IEEE, 2019, pp. 1–6.
\bibitem{b6} Podder, N.K., et al., 2020. Molecular relationships in COVID-19 comorbidities. Genomics, 112(4), pp. 301-310.
\bibitem{b7} Datta, R., et al., 2020. Genetic connections in gastric cancer. Cancer Research, 80(6), pp. 145-153.
\bibitem{b8} Podder, N.K., et al., 2022. Influential genes in glioblastoma. TCGA Analysis, 45(2), pp. 89-96.
\bibitem{b9} Podder, N.K., et al., 2022. Genes in welding fume and respiratory diseases. Respiratory Medicine, 78(3), pp. 112-120.
\bibitem{b10} Rana, M.S., et al., 2023. Genetic links in Parkinson’s and neurodegenerative diseases. Neurology, 95(4), pp. 201-209.
\bibitem{b11} S. Siddique, S. Islam, E. E. Neon, T. Sabbir, I. T. Naheen, and R. Khan, "Deep learning-based Bangla sign language detection with an edge device," Intelligent Systems with Applications, vol. 18, p. 200224, 2023.
\bibitem{b12} Miah, A. S. M., Shin, J., Hasan, M. A. H., \& Rahim, M. A. (2022). BenSignNet: Bengali sign language alphabet recognition using concatenated segmentation and convolutional neural network. Applied Sciences, 12.
\bibitem{b13} Durrani, I.A., et al., 2023. Bidirectional connections in T2D and breast cancer. Breast Cancer Research, 25(3), pp. 78-85.
\bibitem{b14} Talihati, Z., et al., 2025. DEGs in colorectal cancer and comorbidities. Bioinformatics Advances, 5(1), pp. 45-53.
\bibitem{b15} LeCun, Yann; Bengio, Yoshua; Hinton, Geoffrey (2015). \href{https://hal.science/hal-04206682/file/Lecun2015.pdf} {Deep learning}
\bibitem{b16} Ciresan, D.; Meier, U.; Schmidhuber, J. (2012). Multi-column deep neural networks for image classification. 2012 IEEE Conference on Computer Vision and Pattern Recognition.
\bibitem{b17} Krizhevsky, Alex; Sutskever, Ilya; Hinton, Geoffrey (2012). ImageNet Classification with Deep Convolutional Neural Networks. NIPS 2012.
\bibitem{b18} B. P. Amiruddin and R. E. A. Kadir, “Cnn architectures performance evaluation for image classification of mosquito in indonesia,” in 2020 International Seminar on Intelligent Technology and Its Applications (ISITIA), pp. 223–227, IEEE, 2020.
\bibitem{b19} Barua, J.D., et al., 2022. Risk factors in cardiovascular disease progression. Cardiovascular Research, 118(4), pp. 301-310.
\bibitem{b20} Krizhevsky, Alex, Ilya Sutskever, and Geoffrey E. Hinton. "ImageNet Classification with Deep Convolutional Neural Networks". Advances in Neural Information Processing Systems, 2012.
\bibitem{b21} Ian Goodfellow, Yoshua Bengio, and Aaron Courville. "Deep Learning". MIT Press, 2016.
\bibitem{b22} Jordan, M. I., and Mitchell, T. M. (2015). "Machine learning: Trends, perspectives, and prospects."
\bibitem{b23} "Hands-On Machine Learning with Scikit-Learn, Keras, and TensorFlow" by Aur\'elien G\'eron.
\bibitem{b24} Singh, A., \& Kaur, L. (2019). "Comprehensive review on deep learning-based methods for image classification."
\bibitem{b25} Rahman, M.H., et al., 2023. Bayesian model for GEO datasets. Journal of Bioinformatics, 39(2), pp. 67-74.
\bibitem{b26} Smith, J., et al., 2024. Multi-omics analysis of cardiovascular complications in beta-thalassemia. Journal of Bioinformatics, 45(3), pp. 123-135.
\bibitem{b27} Gupta, R., et al., 2023. Prevalence of endocrine complications in beta-thalassemia major patients in India. Clinical Endocrinology, 78(4), pp. 456-463.
\bibitem{b28} Khan, A., et al., 2023. Comparative genomic analysis of beta-thalassemia and sickle cell anemia. Blood Research, 58(3), pp. 112-120.
\bibitem{b29} Zhang, Y., et al., 2023. Bayesian framework for multi-disease genomic analysis. Journal of Computational Biology, 30(5), pp. 210-222.
\bibitem{b30} Lee, K., et al., 2024. Improved DESeq2 for RNA-seq analysis in hematological disorders. Bioinformatics, 40(2), pp. 89-97.
\bibitem{b31} Rossi, M., et al., 2024. CRISPR-Cas9 gene therapy for beta-thalassemia. New England Journal of Medicine, 390(7), pp. 601-610.
\bibitem{b32} Patel, S., et al., 2024. Repurposing metformin for iron overload in beta-thalassemia. Journal of Hematology, 29(2), pp. 77-85.
\bibitem{b33} Kim, H., et al., 2023. CNN-based prediction of cardiomyopathy in beta-thalassemia using cardiac MRI. Medical Imaging Journal, 15(4), pp. 321-330.
\end{thebibliography}

\end{document}