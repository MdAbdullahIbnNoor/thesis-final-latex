%
% File: chap01.tex
% Author: Victor F. Brena-Medina
% Description: Introduction chapter where the biology goes.
% 
\let\textcircled=\pgftextcircled
\chapter{Conclusion and Future Work}
\label{chap:result}
In this chapter, the conclusion and future work have been clarified. For discussion convenience, there are a total of 2 sub-chapters under Chapter 5, which we introduced. In Section 5.1, we discussed the conclusion part; in Section 5.2, we discussed our thesis future work.

\vspace{150mm}
%\vspace{2mm}

%=======
\section{Discussion}
This study has conducted a thorough genetic profiling analysis to unravel the molecular connections between beta-thalassemia and its comorbidities including cardiac complications such as arrhythmogenic cardiomyopathy (ACM), arrhythmia and endocrine complications such as polycystic ovary syndrome (PCOS), type 2 diabetes (T2D), hypothyroidism, and hypogonadism by using Gene Expression Omnibus (GEO) datasets from NCBI. By identifying differentially expressed genes (DEGs), we uncovered shared dysregulated genes across these conditions, and then visualized these DEGs through heatmap representation which highlight common molecular mechanisms driven by iron overload. Our analysis of signaling pathways, Gene Ontology with GO terms and Human Phenotype Ontology (HPO) terms, alongside Protein-Protein Interaction (PPI) and Protein-Drug Interaction (PDI) networks, reveals a robust genetic and molecular linkage between beta-thalassemia and its comorbidities. The PPI networks was constructed using the STRING database, identified hub genes like CD74, BCL2 and FN1 as central players in inflammation, apoptosis, and extracellular matrix remodeling. While PDI networks identify potential drug targets such as C1QA and PLA2G7 for anti-inflammatory and metabolic therapies. Phylogenetic analysis based on FASTA sequences from NCBI offers evolutionary insights into the associative relationships and suggesting a stronger ties between T2D-arrhythmia, hypothyroidism-PCOS and the main strong connections between BT-hypogonadism. These findings are validated against gold-standard databases like dbGaP and OMIM, reinforcing the genetic association and providing a solid foundation for understanding the interplay of these conditions. The identified pathways underscore the role of iron-induced oxidative stress and inflammation in comorbidity progression. The PDI results open avenues for developing targeted treatments, potentially reducing the burden of cardiac and endocrine complications. \\


Overall, this investigation deepens the molecular understanding of beta-thalassemia’s systemic effects, offering valuable insights for designing diagnostic tools and therapeutic strategies. By targeting shared molecular pathways and leveraging the identified drug interactions, this work supports precision medicine approaches and advances systems biology, paving the way for improved management and treatment of beta-thalassemia and its associated conditions.

\section{Future Work}
This study has focused on the molecular relationship between beta-thalassemia and its comorbidities. However, the impact of iron overload and related dysregulated pathways may extend to additional conditions beyond those explored here. Future research can broaden the scope to include other potential comorbidities, such as liver fibrosis or osteoporosis or its potential cancer diseases with survival analysis, which are also influenced by iron metabolism. To deepen our insights, more advanced methodologies, including cutting-edge bioinformatics tools, can be employed to pinpoint critical disease markers. Given the global significance of conditions linked to iron overload, like beta-thalassemia, further investigation into its diverse manifestations is essential. Analyzing larger and more diverse datasets this study will integrating genetic and clinical information by using Artificial Intelligence and Machine Learning techniques which will enhance our ability to predict disease outcomes and tailor personalized treatment strategies.

\section{Conclusion}

Beta-thalassemia, a prevalent genetic disorder, significantly influences the development of its comorbidities such as which is very necessitating greater awareness of these interconnected health challenges. Through the analysis of common differentially expressed genes (DEGs), signaling pathways, Gene Ontology (GO) and Human Phenotype Ontology (HPO) terms, Protein-Protein Interaction (PPI) networks, and Protein-Drug Interaction (PDI) networks, this study has established a robust genetic and molecular relationship between beta-thalassemia and its comorbidities. We identified sufficient dysregulated genes, such as CD74, BCL2, and FN1, which underscore a strong association and heightened risk factor across these conditions that supported by PPI hubs and PDI targets like C1QA and PLA2G7. Our pathway analyses, including T-helper Cell Differentiation and Reverse Cholesterol Transport elucidated critical biological processes, cellular functions, and molecular mechanisms, offering vital insights into disease progression and potential therapeutic targets driven by iron overload.\\


Validation through gold-standard databases like dbGaP and OMIM, combined with phylogenetic insights reinforces the genetic linkage. Understanding these biological connections enables the identification of individuals at higher risk for comorbidities, facilitating early detection, timely intervention, and increased awareness, ultimately improving management and treatment strategies for beta-thalassemia and its associated conditions.