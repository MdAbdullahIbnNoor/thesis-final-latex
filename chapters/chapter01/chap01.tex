%
% File: chap01.tex
% Author: Victor F. Brena-Medina
% Description: Introduction chapter where the biology goes.
%
\let\textcircled=\pgftextcircled
% Remove any custom redefinition of \section or conflicting packages in your main tex file or preamble.
\chapter{Introduction}
\label{chap:intro}

This chapter presents an overview of our research thesis, organized into seven sub-chapters for clarity. In section 1.1, we introduce the topic, focusing on the interplay between diabetes and cancers (breast, ovarian, cervical, and gastric). Section 1.2 outlines the motivation, emphasizing the need to understand shared biological pathways. Section 1.3 defines the problem statement, addressing the increased cancer risk in diabetic patients. Section 1.4 specifies the objectives, including identifying common differentially expressed genes (DEGs) and analyzing molecular interactions. Section 1.5 highlights the research outcomes and their potential impact on early diagnosis and treatment. Section 1.6 details the thesis structure, and section 1.7 provides a discussion on the significance and relevance of our study.

\vspace{200mm}

%=======
\section{General Introduction}
\label{sec:sec1_1}
Beta-thalassemia is a complex hematological disorder inherited from parents to offspring. It is brought by mutations in the HBB gene and caused by the body's inability to produce enough beta-globin protein in red blood cells, which results in chronic anemia, iron overload and several medical complications. Beta-thalassemia patients require lifetime medication and often need regular blood transfusions to manage chronic anemia \cite{b1}.

Some compelling research evidence shows it contributes remarkably to the development of cardiovascular diseases like ACM and arrhythmia, as well as endocrine diseases like T2D, PCOS, hypothyroidism and hypogonadism. Through gene expression analysis, this study aims to explore the link between beta-thalassemia and its associated comorbidities. Many studies show that iron overload from blood transfusions in beta-thalassemia interrupts the function of endocrine glands and reduces hormone synthesis and secretion \cite{b1}. Also, chronic anemia and iron overload from regular blood transfusion cause cardiac abnormalities \cite{b2}.

To explore genetic links more deeply between beta-thalassemia and comorbidities, this study identified differentially expressed genes (DEGs) from RNA-sequence and microarray datasets. This study constructed disease-gene association networks (DGNs) by mapping the genes of up-regulated and down-regulated and then visualizing those matching genes using a heatmap \cite{b3}. Pathway analysis is conducted with graphical representation and gene ontology (GO) analysis to show molecular functions \cite{b3}. This work built the PPI and PDI networks to investigate the functional connectivity between matching genes. A phylogenetic analysis is performed to observe the interrelation of all seven diseases. Finally, this thesis validated the results using standard biomedical libraries like OMIM (Online Mendelian Inheritance in Man) and databases like dbGaP. Our proposed work has been undertaken to achieve a more comprehensive genetic understanding of beta-thalassemia with its comorbidities, which will improve the clinical approaches for its diagnosis, management and treatment. \\

\textbf{Beta-Thalassemia} \\
When the ($\beta^{+}$) or ($\beta^{0}$) synthesis is reduced in hemoglobin of beta globin chains, the $\beta$-thalassemia blood genetic disorder occurs in patients \cite{b4}. It manifests in three forms:
\begin{itemize}
    \item The carrier state,
    \item Thalassemia intermediate (mild to severe), and
    \item Thalassemia major (severe, transfusion-dependent anemia).
\end{itemize} 
\vspace{5mm}
\textbf{Endocrine Disorder} \\
An endocrine disorder is a condition resulting from abnormalities in hormone production, regulation, or action by the endocrine glands, such as the thyroid or pancreas. It includes hypothyroidism and hyperthyroidism, which can cause significant mental health changes. Diabetes mellitus, marked by hyperglycaemia due to insulin issues, is another example. These disorders often lead to systemic effects, including psychiatric symptoms, metabolic imbalances, and increased risk of comorbidities like cardiovascular diseases \cite{b5}. \\

\textbf{Cardiovascular Disorder} \\
A cardiovascular disorder, broadly referred to as heart disease or cardiovascular disease, is any condition that negatively affects the structure or function of the heart and its associated blood vessels \cite{b6}. \\


\section{General Background}
\label{sec:sec1_2}
Beta-thalassemia is an inherited blood disorder transferred from parents in a received patterns that is characterized by absence of beta-globin production, which resulting in anemia, excessive iron accumulation and various complications affecting multiple organ systems. Over time, excessive iron deposition in endocrine glands contributes to a spectrum of complications collectively known as thalassemic endocrine disease (TED). These include hypogonadism, hypothyroidism, diabetes, adrenal dysfunction, and reduced bone mineral density \cite{b7}. It also resulting in various cardiovascular issues such as vasculopathies, cardiomyopathy, hypertension and arrhythmias. These complications significantly contribute to the risk of sudden cardiac death \cite{b8}. 

Although from the current existing research much is known about beta-thalassemia and its complications, the genetic and molecular links between BT and its endocrine and cardiovascular comorbidities remain unclear. Understanding of the shared genes, pathways, PPI and PDI in still limited. 

This study aims to fill these gaps by analyzing gene expression data to uncover the genetic connections between BT and its related complications. This findings could improve the knowledge of disease mechanisms and help to create better diagnostic tools and therapies.
 
\section{Motivation}
\label{sec:sec1_3}
As BT is a chronic inherited disorder that often leads to serious endocrine and cardiovascular complications. Although clinical management has improved day by day but the genetic and molecular perspectives leading these comorbidities are not still clearly understandable. The lack of knowing about internal genetic and molecular relationships limits the ability to design appropriate diagnostic ways and proper treatments.

By analyzing deeply gene expression datasets and using advanced bioinformatics methods, this work provides an opportunity to investigate the shared genetic pathways, GO analysis and gene interaction networks BT and its comorbidities. This study aims to contribute some opinions to the development of more effective diagnostic approaches and treatment strategies for BT patients, ultimately improving their quality of life and reducing the burden of associated complications.

By identifying key genes and molecular mechanisms, this study aims to contribute to the development of more effective diagnostic biomarkers, therapeutic goals and treatment strategies that obviously improve outcomes and quality of health for patients.

\section{Problem Statement}
\label{sec:sec1_4}
Mutations in the HBB gene of blood causing the hereditary disorder named Beta-thalassemia (BT) because of the reduced production of hemoglobin. Hemoglobin is one type of protein found in red blood cells that contains iron and transfers oxygen to tissues across the body. Patients with BT have decreased hemoglobin levels that leads to a reduced supply of oxygen throughout the body.  As a result, patients have anemia which requires regular blood transfusions for life and causes weakness, skin problems, fatigue and sometimes serious health issues. Thalassemia major and thalassemia intermediate are the two types of BT where BT major patient's dependent on regular blood transfusions \cite{b9} and BT minor patient's need not to blood transfusion. Although the blood-related symptoms of BT are well established, increasing evidence shows that the disease is linked to various serious comorbid conditions. These include endocrine diseases such as T2D, PCOS hypothyroidism and hypogonadism as well as cardiovascular complications such as ACM and arrhythmia. Such comorbidities significantly worsen patient prognosis, reduce quality of life, and increase mortality rates. These complications produced negative prognosis, life with health issues and increase the risk of death.

Despite extensive clinical research, the proper molecular and genetic mechanisms that link the BT to these comorbidities remain poorly understood. Many Current studies focus on individual complications separately without exploring the potential shared genetic pathways or overlapping disease mechanisms. These approaches limit the identification of common genes and molecular perspectives that limit utilization of early diagnosis, prediction of future risk and specific treatment. Furthermore, most available research is clinical or descriptive lacking of integrative bioinformatics approaches. Where this work can analyze large-scale genomic data to uncover underlying relationships between BT and its associated disorders.

Facing several key challenges because of the absence of such integrative approaches. Firstly, it is difficult to develop diagnostic tools that can detect comorbidities at an early stage without identifying the shared genes and molecular relationships. Secondly, the lack of molecular insights affects the development of targeted therapies that address simultaneously both BT and its comorbidities rather than treating them as separated. Finally, the opportunities for drug recycling or the development of therapeutic approaches will be missing without mapping DGNs, PDI and PPI networks.

Addressing these gaps requires a systematic and computational approach that can integrate gene expression datasets from multiple diseases to identify overlapping genes, pathways, and protein interactions. In this study, publicly available microarray and mRNA datasets from NCBI are analyzed to perform genetic profiling of BT with its major comorbidities. The analysis involves constructing DGNs, heatmap visualization of gene patterns, pathway and ontology, PPI and PDI mapping as well as phylogenetic analysis to investigate the evolutionary connections between the diseases.

This research aims to overcome these knowledge gaps, enabling a integrative medical strategies that can improve the ways of diagnosis, treatment design and ultimately enhance the outcomes of patient situation by revealing the shared genetic relationships between BT and its comorbidities.


\section{Thesis Objectives}
\label{sec:sec01}

The primary objectives of this research is to investigate the genetic relationships between BT and its selected endocrine and cardiovascular comorbidities. This work aims to identifying the mechanisms of shared molecular that may help in integrative diagnostic and therapeutic approaches.

The specific objectives are:


\begin{enumerate}[label=\arabic*)]
\item Identify the common differentially expressed genes (DEGs) between BT and its associated comorbidities including hypothyroidism, hypogonadism, PCOS, T2D, ACM and arrhythmia by using microarray and mRNA datasets from publicly available NCBI tools.
\item Construct and visualize disease-gene networks (DGNs) to show the genetic overlaps between BT and each comorbidities.
\item Visualize heatmap analysis to reveal the gene expression patterns across BT and the selected comorbidities.
\item To explore the biological processes, molecular functions and cellular components of the shared genes pathway and gene ontology enrichment analysis is conducted.
\item Build protein-protein interaction (PPI) networks to identify the key hub proteins from molecular interaction.
\item Map protein-drug interactions (PDI) to highlight potential therapeutic treatments and opportunities for drug recreations.
\item Create a validation network to confirm that the selected comorbidities are valid for the beta-thalassemia based on genetic perspectives.
\item To determine the evolutionary relationships among BT and its comorbidities phylogenetic analysis is performed based on the shared genes.
\end{enumerate}


This study aims to bridge the gap between genetic profiling and molecular interactions that helps in clinical application, supporting early detection of health risk, personalized treatment and improved patient outcomes affected by BT and its related comorbidities.


\section{Thesis Contribution}
\label{sec:sec02}

The proposed computational framework provides an integrated bioinformatics approach to exploring the genetic connections of beta-thalassemia and its associated comorbidities. By analyzing microarray and mRNA datasets from National Center for Biotechnology Information NCBI, this study identifies shared genes, molecular pathways and uncovers potential therapeutic targets. 

The main contributions of this research are:


\begin{enumerate}[label=\arabic*)]
\item A total of 15,008 differentially expressed raw genes were analyzed from NCBI platforms that identifying significantly expressed genes associated with beta-thalassemia and its comorbidities.
\item A total of six comorbidities have been selected to validate the connections with BT using gold benchmark.
\item This study discovered 13, 165, 13, 14, 11, and 44 common DEGs for hypothyroidism, hypogonadism, PCOS, T2D, ACM and arrhythmia respectively that highlighting the potential molecular links between beta-thalassemia and these conditions.
\item Built and analyzed DGN networks to visualize genetic interconnections and identify key hub genes that playing a central role in disease progression.
\item Revealed relevant signaling pathways and Gene Ontology terms strongly associated with the shared DEGs.
\item Conducted PPI and PDI analyses to identify the central proteins and the potential drug molecules that may target both beta-thalassemia and its comorbidities.
\item Developed a validation network to validate the selection of comorbidities and then performed phylogenetic analysis to determine the evolutionary relationships among the diseases.
\item Provides an established view of how genetic, molecular and evolutionary factors interplay in beta-thalassemia and its associated complications to providing the foundation of more targeted diagnostics and personalized treatment strategies.
\end{enumerate}

\section{Significance}
\label{sec:sec03}

This study has the substantial importance in advancing the understanding of beta-thalassemia and its associated comorbidities through an integrative genetic and bioinformatics approach. By identifying key genetic mutations, enriched pathways, functional ontologies and regulatory mRNAs, this study aims to pave the way for identifying potential biomarkers that could improve early diagnosis and help predict disease outcomes. Exploring the networks of protein-protein and protein-drug interaction will help in the identification of novel therapeutic targets and support the opportunities of drug repurposing.

Furthermore, phylogenetic analysis of disease-associated mutations will provide evolutionary insights into their origin and prevalence that enhancing the global understanding of epidemiological. Overall, the findings of this study are expected to support more personalized treatment strategies, guide future molecular research and increase the awareness about the complex health risks faced by beta-thalassemia patients.

\section{Thesis Layout}
\label{sec:sec04}

The rest of the thesis is organized as follows.

\vspace{0.5cm}
\textbf{CHAPTER 1:}
This chapter introduces the background of beta-thalassemia, its clinical significance and the motivation for this study. It also outlines the problem statement, objectives, scope, goal and expected contributions of the research.

\vspace{0.2cm}

\textbf{CHAPTER 2:}
This chapter presents a comprehensive literature review, summarizing previous studies from reputable journals and conferences. It highlights the current understanding of shared molecular mechanisms and identifies the research gaps.

\vspace{0.2cm}

\textbf{CHAPTER 3:}
This chapter describes the methodology and datasets used in the study. It details the sources of microarray and mRNA data obtained from NCBI, the preprocessing steps, and the computational approaches applied such as the Benjamini Hochberg for identifying shared genes. Another method is used for diseasome network construction, pathway and ontology analysis, protein-protein and protein-drug interaction mapping and phylogenetic analysis.

\vspace{0.2cm}

\textbf{CHAPTER 4:}
The most significant chapter that presents the results of the analyses, including the identification of common differentially expressed genes between beta-thalassemia and each comorbidity, visualization of gene expression patterns and functional enrichment findings. It also reports the construction of interaction networks, key hub proteins, drug association analysis, and evolutionary relationships.

\vspace{0.2cm}

\textbf{CHAPTER 5:} 
This chapter concludes the thesis by summarizing the major findings and their implications for clinical research and treatment strategies. It also outlines the limitations of the study and proposes directions for future work in exploring genetic and molecular connections between beta-thalassemia and other diseases.

\vspace{0.2cm}

\
\section{Conclusion}
\label{sec:sec1_5}

\space This introductory chapter describes the main research questions, goals, objectives and the motivation of driving this study. The subsequent chapters will present a detailed review of existing literature, describe the methodology for data collection, data analysis, and discuss the results in relation to the current knowledge. The next chapters will go over each of them in depth.
%=========================================================